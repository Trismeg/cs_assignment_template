\documentclass{tufte-handout}

\title{An Assignment Template}

\author[The Academy]{Sam Student}

%\date{28 March 2010} % without \date command, current date is supplied

%\geometry{showframe} % display margins for debugging page layout

\usepackage{graphicx} % allow embedded images
  \setkeys{Gin}{width=\linewidth,totalheight=\textheight,keepaspectratio}
  \graphicspath{{graphics/}} % set of paths to search for images
\usepackage{amsmath}  % extended mathematics
\usepackage{booktabs} % book-quality tables
\usepackage{units}    % non-stacked fractions and better unit spacing
\usepackage{multicol} % multiple column layout facilities
\usepackage{lipsum}   % filler text
\usepackage{fancyvrb} % extended verbatim environments
  \fvset{fontsize=\normalsize}% default font size for fancy-verbatim environments
  
  
  
    %MADNESS
  
  \usepackage[T1]{fontenc} % Use 8-bit encoding that has 256 glyphs
\usepackage{fourier} % Use the Adobe Utopia font for the document - comment this line to return to the LaTeX default
\usepackage[english]{babel} % English language/hyphenation
\usepackage{amsmath,amsfonts,amsthm} % Math packages
\usepackage{mathtools}% http://ctan.org/pkg/mathtools
\usepackage{etoolbox}% http://ctan.org/pkg/etoolbox
\usepackage{lipsum} % Used for inserting dummy 'Lorem ipsum' text into the template
\usepackage{units}% To use \nicefrac
\usepackage{cancel}% To use \cancel
\usepackage{physymb}%To use r
\usepackage{sectsty} % Allows customizing section commands
\usepackage[dvipsnames]{xcolor}
\usepackage{pgf,tikz}%To draw 
\usepackage{pgfplots}%To draw 
\usetikzlibrary{shapes,arrows}%To draw 
\usetikzlibrary{patterns,fadings}
 \usetikzlibrary{decorations.pathreplacing}%To draw curly braces 
 \usetikzlibrary{snakes}%To draw 
 \usetikzlibrary{spy}%To do zoom-in
 \usepackage{setspace}%Set margins and such
 %\usepackage{3dplot}%To draw in 3D
\usepackage{framed}%To get shade behind text


\definecolor{shadecolor}{rgb}{0.9,0.9,0.9}%setting shade color
\allsectionsfont{\centering \normalfont\scshape} % Make all sections centered, the default font and small caps
  
  

  
  

% Standardize command font styles and environments
\newcommand{\doccmd}[1]{\texttt{\textbackslash#1}}% command name -- adds backslash automatically
\newcommand{\docopt}[1]{\ensuremath{\langle}\textrm{\textit{#1}}\ensuremath{\rangle}}% optional command argument
\newcommand{\docarg}[1]{\textrm{\textit{#1}}}% (required) command argument
\newcommand{\docenv}[1]{\textsf{#1}}% environment name
\newcommand{\docpkg}[1]{\texttt{#1}}% package name
\newcommand{\doccls}[1]{\texttt{#1}}% document class name
\newcommand{\docclsopt}[1]{\texttt{#1}}% document class option name
\newenvironment{docspec}{\begin{quote}\noindent}{\end{quote}}% command specification environment

\begin{document}

\maketitle% this prints the handout title, author, and date
\begin{marginfigure}%
  \includegraphics[width=\linewidth]{IBM.jpg}
  \caption{This is a margin figure.  Here is where you put the caption for your margin figure.}
  \label{fig:marginfig}
\end{marginfigure}
\begin{abstract}
\noindent
This is the abstract for the assignment.  Here you state what the assignment is and provide any additional introduction or commentary that you care to.
\end{abstract}

\normalsize

%this generates 1cm of vertical space
\vspace{1cm}
\section{Command Line Stuff}

This is how you can represent things you are entering into the command line terminal.  Note the use of the dollar sign like the terminal has.

\marginnote[30pt]{This is a margin note you can use to comment on what you are doing in the command line.}

\begin{shaded}
\begin{verbatim}
$ git pull -u origin master
\end{verbatim}
\end{shaded}

\vspace{1cm}

\section{Representing Python Code in Your Assignment}

\marginnote[40pt]{Here you can comment on your python code.}

\begin{framed}
\begin{verbatim}
a=21
b="prime"

for x in range(2,a/2+1):
    if a%x==0:
        b="composite"
        break
print "the number is "+b
\end{verbatim}
\end{framed}

\marginnote[40pt]{This indicates the output of your program.}

\begin{shaded}
\begin{verbatim}
the number is composite
\end{verbatim}
\end{shaded}

\bibliography{sample-handout}
\bibliographystyle{plainnat}



\end{document}
